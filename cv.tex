\documentclass[a4paper]{article}
\usepackage[utf8]{inputenc}
\usepackage[T1]{fontenc}
\usepackage{marvosym}
\usepackage{simplemargins}

\usepackage{ifthen}
\ifdefined\doen
\usepackage[english]{babel}
\else
\usepackage[francais]{babel}
\fi
\newcommand\enfr[2]{\ifdefined\doen#1\else#2\fi\xspace}
\usepackage{array}
\newcolumntype{x}[1]{>{\raggedleft\hspace{0pt}}p{#1}}

\setleftmargin{1cm}
\setrightmargin{1cm}
\settopmargin{1cm}
\setbottommargin{1cm}
\pagestyle{empty}
\usepackage{paralist}
\usepackage{url}
\usepackage{xspace}
\usepackage[svgnames]{xcolor}
\usepackage{amsmath}
\usepackage{xfrac}
\usepackage{hyperref}
\usepackage{bibtopic}
\makeatletter
\renewcommand\@biblabel[1]{\vspace{1em}\hspace{-0.5em}}
\makeatother


\newcommand\skillset[2]{
  \noindent\begin{tabular}{x{0.15\textwidth}|p{0.83\textwidth}}
     #1 & #2 \\
  \end{tabular}
  \vspace{0.2cm}
}

\newcommand\cpplang{\textbf{C\raisebox{0.1em}{++}}\xspace}
\newcommand\php{\textbf{PHP}\xspace}
\newcommand\js{\textbf{Javascript}\xspace}
\newcommand\sql{\textbf{SQL}\xspace}
\newcommand\mpi{\hyperlink{https://en.wikipedia.org/wiki/Message_Passing_Interface}{\textbf{MPI}}\xspace}
\newcommand\clang{\textbf{C}\xspace}
\newcommand\cmake{\textbf{CMake}\xspace}
\newcommand\rlang{\textbf{R}\xspace}
\newcommand\perl{\textbf{Perl}\xspace}
\newcommand\python{\textbf{python}\xspace}
\newcommand\bash{\textbf{Bash}\xspace}
\newcommand\java{\textbf{Java}\xspace}
\newcommand\ada{\textbf{Ada}\xspace}
\newcommand\blang{\hyperlink{https://www.methode-b.com/en/b-method/}{\textbf{B method}}\xspace}
\newcommand\html{\textbf{HTML}\xspace}
\newcommand\lisp{\textbf{lisp}\xspace}

\newcommand\heading[1]{\Large\textsf{\textbf{\textcolor{DodgerBlue}{#1}}}\normalsize}
\newcommand\entry[2]{\large\textcolor{Black}{\textbf{#1} --- #2}\normalsize}
\newcommand\sideentry[1]{\normalsize\textbf{#1}~~}
\newcommand\timespan[2]{\normalsize\textsf{#1 -- #2}}

\hypersetup {
  colorlinks=true,
  urlcolor=DodgerBlue,
  linkcolor=DodgerBlue,
  citecolor=DodgerBlue,
}

\begin{document}
\color{DarkSlateGray}

%\defaultleftmargin{0em}{1em}{1em}{1em}.

\begin{minipage}[t]{0.63\textwidth}
  \Huge\textcolor{Black}{\textbf{Clément Rezvoy}}\normalsize\\

  PhD in Computer Science\\
  Senior Software Developper\\
  Junior Team Manager\\
  Rookie Product Owner\\
\end{minipage}
\hfill
\begin{minipage}[t]{0.27\textwidth}
  \hyperlink{tel:+33-675-633-563}{\enfr{+33 675 633 563}{06 75 63 35 63}} \\
  \hyperlink{mailto:clement.rezvoy@gmail.com}{clement.rezvoy@gmail.com} \\
  \hyperlink{https://linkedin.com/in/crezvoy}{linkedin.com/in/crezvoy} \\
  \hyperlink{https://github.com/crezvoy}{github.com/crezvoy}
\end{minipage}

\begin{minipage}[t]{0.63\textwidth}
\heading{\enfr{Work Experience}{Expérience Professionnelle}}\\

\entry{Cosmo Tech, Lyon\enfr{ FR}}{\enfr{Software Engineer, Team Lead, and Product Owner}{Développeur, \textit{Team Lead}, et \textit{Product Owner}}}\\
\textit{\enfr{Cosmo Tech is a global software vendor of Enterprise Digital Twin
    solutions for the industry to simulate and optimize operational efficiency.}
  {Cosmo Tech est un éditeur de logiciels fournisssant des solutions de jumeaux numériques pour l'industrie afin de simuler et optimiser l’efficacité opérationnelle}}

\vspace{0.5em}
\timespan{\enfr{May}{Mai} 2014}{\enfr{Feb}{Fev}. 2017}
\begin{compactitem}
\item[\textbullet] \enfr{Entered Cosmo Tech (\textit{née} The CoSMo Company)
  as a Software Engineer, Working on the complex system modeling
  language and simulation runtime, both written mainly in \cpplang
  with bindings in \java and \python}
    {Débuts chez Cosmo Tech (alors The CoSMo Company) comme
      Développeur, travaillant sur le langage de modélisation de
      systèmes complexes et l’environnement d'exécution des
      simulations, tous deux écrits principalement en \cpplang avec
      des surcouches en \java et \python};
\item[\textbullet] \enfr{introduced concurrency into the simulation
  scheduler to allow multithreaded simulations}
  {Introduction de la notion de concurrence dans le \textit{scheduler}
    de la simulation, permettant la parallélisation de certaines
    étapes de la simulation};
\item[\textbullet] \enfr{wrote a build and artifact management system
  for third party libraries and tools based on \cmake}
  {Création d'un système de build et de gestion des artéfacts basé sur
    \cmake pour la gestion de nos dépendances \textit{open source}};
\item[\textbullet] \enfr{introduced the notion of project and
  dependencies to better handle project life cycle and allow for
  project composability}
  {Formalisation la notion de projet et de dépendances pour permettre
    de mieux gérer le cycle de vie des projets et permettre les liens
    de composition entre projets}.
\end{compactitem}

\vspace{0.5em}
\timespan{\enfr{Feb}{Fev}. 2017}{\enfr{Apr}{Avr}. 2019}
\begin{compactitem}
\item[\textbullet] \enfr{In addition, I became \textbf{Team Lead} for my team (5
software engineers). The position grouped the responsibility of
both \textbf{scrum master} and \textbf{manager}}
  {\textit{Team Lead} pour mon équipe (5 Ingénieurs). Ce poste
    regroupe les responsabilité de \textit{\textbf{scrum master}} et
    de \textit{\textbf{manager}}};
\item[\textbullet] \enfr{Organized the day to day life the team, daily
  stand-ups, retrospectives. Making sure the workload was
  appropriately assigned to the members of the team and that the team
  reached its commitment by the end of the sprint}
  {Organisation de la vie de l'équipe au jour le jour,
    \textit{standups} et rétrospectives. Je m'assurais que le travail
    était correctement distribué au sein de l'équipe et que l'équipe
    tenait ses engagements de fin de \textit{sprint}};
\item[\textbullet] \enfr{Worked with the product owner to organize and
  facilitate reviews and backlog groomings}
  {Organisation et facilitation des \textit{feature reviews} et des
    \textit{backlog grooming} en coordination avec le \textit{product
      owner}};
\item[\textbullet] \enfr{management responsibilities, including,
  weekly one-on one, bi-annual performance reviews, hiring interviews,
  and synchronization with other teams}
  {Responsabilités de \textit{management}, réunions individuelles de
    suivi, revues des objectifs semestrielles, recrutement et
    synchronisation avec les autres équipes}.
\end{compactitem}

\vspace{0.5em}
\timespan{\enfr{Apr}{Avr}. 2019}{\enfr{Present}{Présent}}
\begin{compactitem}
\item[\textbullet] \enfr{Currently holding the role of Cosmo Tech's platform \textbf{product
  owner}; platform encompasses the modeling language, the
    simulation runtime and deployment tools}
  {J'occupe actuellement le poste de \textit{Product Owner} de la plateforme de Cosmo Tech; la plate-forme regroupe le langage de modélisation, l’environnement de simulation et les outils de déploiement};
\item[\textbullet] \enfr{I am in charge of the product backlog, working with
  the chief product owner, architect and CTO to define the evolution
  of the platform}
  {En charge de la \textit{backlog} du
  produit, Je travaille avec le CTO l'architecte et les autres
  \textit{product owners} à la définition des évolutions de la
  plate-forme};
\item[\textbullet] \enfr{regular presentations of future and current
  developments to company management, users and prospects, grooming
  with dev teams}
  {présentations régulières des développements en
  cours et à venir pour les dirigeants et cadres de la société, les
  utilisateurs et prospects; \textit{grooming} avec les équipes de
  développement pour affiner et estimer les nouvelles fonctionnalités};
\item[\textbullet] \enfr{Functionally validating features, overseeing QA and
  documentation process prior to general availability of new versions}
  {Validation fonctionelles des fonctionalité, supervision des
    campagnes de test et de documentation avant la mise à disposition
    des nouvelles versions}.\\
\end{compactitem}

\entry{Qosmos, Paris\enfr{ FR}{}}{\enfr{Software Engineer}{Développeur}}\\
\textit{\enfr{Enea Qosmos leads the market for embedded IP traffic
    classification and network intelligence technology used in
    physical, SDN and NFV architectures}
  {Enea Qosmos est le leader du marché pour la classification de
    traffic IP embarqué et l'inteligence réseau utilisé dans les
    réseaux physique SDN et NFV}.}

\vspace{0.5em}
\timespan{Oct. 2011}{\enfr{Apr}{Avr}. 2014}
\begin{compactitem}
\item[\textbullet] \enfr{worked on high performance, low level \clang
  code for network optimizations and security}
  {Développement de code haute performance en \clang pour
    l'optimization réseau et la sécurité};
\item[\textbullet] \enfr{reversed engineered and implemented detection
  routines for network protocols and applications for traffic
  classification}
  {Retro-engineering de protocoles réseaux et développement de routines
    de classification pour la classification de traffic};
\item[\textbullet] \enfr{created and led an in depth training program for partners wanting to
  integrate and use Qosmos' products}
  {Création et présentation d'un programme de formation pour des
    partenaires souhaitant s'intégrer avec les produits Qosmos}.
\end{compactitem}
\end{minipage}
\hfill
\begin{minipage}[t]{0.29\textwidth}
  \heading{\enfr{Skills}{Compétences}}\\

  \sideentry{\enfr{Software development}{Développement logiciel}}
  \enfr{I have +10 years of experience designing and writing
    software. I'm independent and can handle and oversee all the
    stages of software development, from initial design to delivery. I
    am mostly accustomed with system-level languages (\clang,
    \cpplang) but I have also written substantial projects with
    scripting languages (\python, \rlang, \bash). I have worked in
    many different environments: distributed systems, near real-time,
    webapps; I am adaptable, I tend to be quick to pickup a new tech
    stack.}
    {J'ai plus de 10 ans d’expérience dans la conception et
    la création de logiciels. Je suis indépendant et je peux
    participer ou gérer toutes les phases du développement logiciel,
    de la demande initiale à la livraison. J'ai plus l'habitude des
    langages systèmes (\clang, \cpplang) mais J'ai également réalisé
    des projets dans des langages de plus haut niveau (\python, \rlang,
    \bash). J'ai travaillé dans des environnement très différents:
    Systèmes distribués, quasi temps-réel, webapps; Je sais m'adapter
    rapidement à un nouvel environnement technique.}\\

    \sideentry{\enfr{Team management}{Gestion d'équipe}}
    \enfr{I can manage a team of developpers
  following and \textbf{agile methodology}; I focus on making sure
  they see the big picture and not only their current ticket.}
    {Je sais gérer une équipe dans un contexte \textbf{agile}. Je
      fait en sorte que les développeurs aient conscience de la
      vision d'ensemble et pas seulement de leur ticket en cours.}\\
 
    \sideentry{\enfr{Product Management}{Gestion de produit}}
    \enfr{I can build from a client
  request or \textbf{from an idea to an actual feature}. I follow the
  initial requirement though all the steps of the process from
  inception to completion and delivery. I make sure alignment between
  management, dev team and users is never lost by communicating
  regularly on the state of the product.}
    {Je sais aller d'une demande client ou \textbf{d'une idée à une
        fonctionnalité livrée}. Je suis la demande initiale à travers
      toutes les étapes du processus de la conception à la réalisation
      et la livraison. Je m'assure de l'alignement entre le
      management, les développeurs et les utilisateurs n'ai jamais
      rompu en communicant régulièrement.}\\

    \sideentry{Communication}
    \enfr{My teaching experience ranged from
  Algorithmics in an elite school to Programming 101 for non computer
  science majors, I learned to \textbf{adapt the message the audience}
  to make sure I am understood.  During my PhD I was the link between
  the Computer Science lab and the genomics lab, I know the importance
  of \textbf{reaching a common understanding} to lead a meaningful
  discussion. Today as a product owner much of my role is making sure
  information flows correctly and losslessly from users to developers
  and back.}
  {Mon expérience d'enseignement m'a appris à \textbf{adapter le
      discours au public} pour être sûr d'être compris. Durant mon
    doctorat, j'étais le lien en le département d'informatique et de
    biologie, Je connais l'importance \textbf{d'arriver à une
      compréhension commune} du problème pour pourvoir en discuter
    efficacement. Aujourd'hui, en tant que \textit{product owner}, Une
    grande partie de mon rôle est de m'assurer que l'information
    circule correctement et sans perte des utilisateurs aux
    développeurs, et inversement.}\\
 
\end{minipage}

\pagebreak
\begin{minipage}[t]{0.63\textwidth}
\heading{\enfr{Academics \& Research}{Formation \& Recherche}}\\

\entry{École Normale Supérieure, Lyon\enfr{ FR}{}}{\enfr{PhD, Computer Science}{Doctorat Informatique}}\\
\textit{\enfr{Dissertation}{Thèse}:} \hyperlink{https://www.theses.fr/en/2011ENSL0643}{\textit{Large Scale Parallel Inference of Protein and Protein Domain families}}

\vspace{0.5em}
\timespan{Sept. 2007}{Sept. 2011}\\
\enfr{I used \cpplang and \mpi to create a program to evenly
distribute the workload of a large scale biological sequence analysis
algorithm across thousands of cpus, while preserving the underlying
biological hypotheses of a non trivially parallel problem. I was
awarded $400,000$ hours of computing time on
\hyperlink{https://www.cines.fr/en/supercomputing-2/}{CINES}
supercomputer and spent countless week-ends and late nights using
\hyperlink{https://www.grid5000.fr/w/Grid5000:Home}{grid'5000}, a
distributed computing infrastructure.}
{J'ai Utiliser de \cpplang et \mpi pour développer un programme
  permettant de répartir efficacement la charge d'une analyse de
  séquences biologiques à grande échelle sur un grand nombre de
  processeurs, tout en préservant les hypothèses biologiques d'un
  problème qui n'était pas trivialement parallèle. On m'a aloué $400
  000$ heures de calcul sur le supercalculateur du
  \hyperlink{https://www.cines.fr/en/supercomputing-2/}{CINES} et j'ai
  également passé de nombreux week-ends et soirées à utiliser
  \hyperlink{https://www.grid5000.fr/w/Grid5000:Home}{grid'5000}, une
  infrastructure de calcul distribué. pour mener à bien ce projet.}

\vspace{0.5em}
\timespan{Sept 2007}{Sept. 2010}\\
\enfr{During my PhD I also worked as a \textbf{teaching assistant} in computer
science, I led recitations and practical sessions in: introduction to
programming, Systems and Networks, Analysis and design of Algorithms.}
{Pendant mon doctorat, j'ai également travaillé comme moniteur de en
  informatique. J'ai dirigé des scéance de TD et TP en introduction à
  la programmation, Système et réseaux et algorithmique.}\\

\entry{Université de Lyon, Lyon\enfr{ FR}{}}{\enfr{MS, Bioinformatics}{Master Bioinformatique}}\\
\textit{\enfr{Master degree in statistical and mathematical modeling of biological
systems}{Master recherche en modélisation statistique et mathématiques des systèmes biologiques.}}

\vspace{0.5em}
\timespan{Sept 2005}{Jun. 2007}\\
\enfr{Completed a Master degree in \textit{Approches Mathématiques et Informatique du
  Vivant}, specializing in sequence analysis. During my first year I
wrote the first version of MareyMap, a tool for estimating meiotic
recombination rate written in \rlang. I published
\hyperlink{https://academic.oup.com/bioinformatics/article/23/16/2188/199692}{a
  paper} about it and I also had the opportunity to present my work in
poster sessions of several french conferences. Sources have since then
been put on \hyperlink{https://github.com/aursiber/MareyMap}{
  github}.}
{J'ai obtenu un Master \textit{Approches Mathématiques et Informatique du
    Vivant}, spécialisé en ananlyse de séquence. Pendant ma première
  année de master, J'ai écris la première version de MareyMap, un
  outils d'estimation du taux de recombinaison meiotique écrit en
  \rlang. J'ai publié \hyperlink{https://academic.oup.com/bioinformatics/article/23/16/2188/199692}{un
  article} à ce propos et j'ai également eu l'occasion de présenter
  mon travail dans plusieurs conférences. Les sources de cet outils on
  depuis été déposées
  \hyperlink{https://github.com/aursiber/MareyMap}{sur github}.}\\

\entry{Teesside University, Middlesbrough UK}{\enfr{BS, Computer Studies}{BSc Computer Studies}}\\
\textit{\enfr{General degree in computer science with a broad range of subjects covered.}
  {Diplôme généraliste en informatique, niveau License, avec un large éventail de sujets abordés}}

\vspace{0.5em}
\timespan{Sept. 2003}{\enfr{June}{Juin} 2005}\\
\enfr{
Modules studied included: object technology (\java), database systems (\sql), networks
\& communications, UI (\html, \js), operating systems (\clang), real time development,
language systems, expert systems, AI (\lisp), safety and high integrity (\ada)
systems, and formal Methods (\blang). I Graduated with a \textbf{First Class
  with Honours}.}{
 Les modules suivis incluaient: programmation objet (\java), base de
 données (\sql), réseaux \& communications, UI (\html, \js), Systèmes
 d'exploitation (\clang), Systèmes temps réel, Langages, Systèmes
 experts, IA (\lisp), Système de haute intégrité (\ada) et méthodes
 formelles (\blang). J'ai validé ce diplôme avec la mention
 \textit{\textbf{First Class with Honours}}.}\\


\entry{Université de Lyon, Lyon\enfr{ FR}{}}{\enfr{AS, Computer Science}{DUT Informatique}}\\
\textit{\enfr{University degree, equivalent of an associate degree in computer science with an
emphasis on software engineering.}{Dut Informatique, option Génie Informatique}}

\vspace{0.5em}
\timespan{Sept 2001}{\enfr{June}{Juin} 2003}\\
\enfr{
Completed a \textit{Diplôme Universitaire de Technologie Informatique
  option génie informatique}. Modules studied included: maths
(algebra, linear algebra, probability \& statistics), algorithms, UML,
programming (\clang), OO-programming (\cpplang, \java), computer
architecture, operating systems, networks, databases (\sql), web dev
(\php, \js). I validated My degree with a 3 month internship at
University of Stafforshire in Stoke On Trent (UK).}
  {Les modules suivis incluaient: Mathématiques (algèbre, algèbre
    linéaire, Probabilités et statistique), algorithmique, UML,
    programmation (\clang), Programmation orientée objet (\cpplang,
    \java), architecture des ordinateurs, Systèmes d'exploitation,
    réseaux, bases de données (\sql), développement web (\php,
    \js). J'ai validé mon diplôme avec un stage de trois mois à
    l'université de Staffordshire à Stoke-on-Trent (UK).}

\end{minipage}
\hfill
\begin{minipage}[t]{0.27\textwidth}
  \heading{\enfr{Presentations}{Présentations}}\\

  \sideentry{JOBIM 2007} \enfr{French bioinformatics conference. Scientific poster titled:}{Poster} \textit{MareyMap : a R-based tool with graphical interface for estimating recombination rates}. \\

  \sideentry{JOBIM 2010} \enfr{Presentation titled:}{Présentation} \textit{Scalability of large-scale protein domain inference}.\\

  \sideentry{ECCB 2010} \enfr{European conference on Computational Biology.}{Conférence européene sur la biologie computationnelle.} \enfr{Presentation titled:}{Présentation} \textit{Scalability of large-scale protein domain family inference}. \\

  \sideentry{\enfr{PhD defense}{Soutenance de doctorat}} September 2011. \\

  \sideentry{2013, Qosmos training} \enfr{3-day in-depth workshop.}{Programme de formation approfondie de 3 jour} \\

  \sideentry{Cosmo Tech Product Roadmap} \enfr{Bi-annual internal
  presentation of the state of the product and its future
  evolutions.}
  {Présentation semestrielle de l'actualité du produit et de ses
    évolutions à venir.}\\

  \begin{btSect}[plain]{mine}
    \heading{\enfr{Papers}{Articles}}
    %\vspace{-1em}
    \btPrintAll
  \end{btSect}
\end{minipage}

\end{document}
